% Options for packages loaded elsewhere
\PassOptionsToPackage{unicode}{hyperref}
\PassOptionsToPackage{hyphens}{url}
%
\documentclass[
  12pt,
]{article}
\usepackage{amsmath,amssymb}
\usepackage{iftex}
\ifPDFTeX
  \usepackage[T1]{fontenc}
  \usepackage[utf8]{inputenc}
  \usepackage{textcomp} % provide euro and other symbols
\else % if luatex or xetex
  \usepackage{unicode-math} % this also loads fontspec
  \defaultfontfeatures{Scale=MatchLowercase}
  \defaultfontfeatures[\rmfamily]{Ligatures=TeX,Scale=1}
\fi
\usepackage{lmodern}
\ifPDFTeX\else
  % xetex/luatex font selection
\fi
% Use upquote if available, for straight quotes in verbatim environments
\IfFileExists{upquote.sty}{\usepackage{upquote}}{}
\IfFileExists{microtype.sty}{% use microtype if available
  \usepackage[]{microtype}
  \UseMicrotypeSet[protrusion]{basicmath} % disable protrusion for tt fonts
}{}
\makeatletter
\@ifundefined{KOMAClassName}{% if non-KOMA class
  \IfFileExists{parskip.sty}{%
    \usepackage{parskip}
  }{% else
    \setlength{\parindent}{0pt}
    \setlength{\parskip}{6pt plus 2pt minus 1pt}}
}{% if KOMA class
  \KOMAoptions{parskip=half}}
\makeatother
\usepackage{xcolor}
\usepackage[margin=1in]{geometry}
\usepackage{longtable,booktabs,array}
\usepackage{calc} % for calculating minipage widths
% Correct order of tables after \paragraph or \subparagraph
\usepackage{etoolbox}
\makeatletter
\patchcmd\longtable{\par}{\if@noskipsec\mbox{}\fi\par}{}{}
\makeatother
% Allow footnotes in longtable head/foot
\IfFileExists{footnotehyper.sty}{\usepackage{footnotehyper}}{\usepackage{footnote}}
\makesavenoteenv{longtable}
\usepackage{graphicx}
\makeatletter
\def\maxwidth{\ifdim\Gin@nat@width>\linewidth\linewidth\else\Gin@nat@width\fi}
\def\maxheight{\ifdim\Gin@nat@height>\textheight\textheight\else\Gin@nat@height\fi}
\makeatother
% Scale images if necessary, so that they will not overflow the page
% margins by default, and it is still possible to overwrite the defaults
% using explicit options in \includegraphics[width, height, ...]{}
\setkeys{Gin}{width=\maxwidth,height=\maxheight,keepaspectratio}
% Set default figure placement to htbp
\makeatletter
\def\fps@figure{htbp}
\makeatother
\setlength{\emergencystretch}{3em} % prevent overfull lines
\providecommand{\tightlist}{%
  \setlength{\itemsep}{0pt}\setlength{\parskip}{0pt}}
\setcounter{secnumdepth}{5}
\usepackage{booktabs}
\usepackage{longtable}
\usepackage{array}
\usepackage{multirow}
\usepackage{wrapfig}
\usepackage{float}
\usepackage{colortbl}
\usepackage{pdflscape}
\usepackage{tabu}
\usepackage{threeparttable}
\usepackage{threeparttablex}
\usepackage[normalem]{ulem}
\usepackage{makecell}
\usepackage{xcolor}
\usepackage{amsmath}
\usepackage{amssymb}
\usepackage{natbib}
\setcitestyle{authoryear,open={(},close={)}}
\usepackage{booktabs}
\usepackage{longtable}
\usepackage{array}
\usepackage{multirow}
\usepackage{wrapfig}
\usepackage{float}
\usepackage{colortbl}
\usepackage{pdflscape}
\usepackage{tabu}
\usepackage{threeparttable}
\usepackage{threeparttablex}
\usepackage[normalem]{ulem}
\usepackage{makecell}
\usepackage{xcolor}
\ifLuaTeX
  \usepackage{selnolig}  % disable illegal ligatures
\fi
\usepackage[]{natbib}
\bibliographystyle{plainnat}
\usepackage{bookmark}
\IfFileExists{xurl.sty}{\usepackage{xurl}}{} % add URL line breaks if available
\urlstyle{same}
\hypersetup{
  pdftitle={Transmisión de precios y ajuste asimétrico al equilibrio: evidencia para las principales ciudades de Colombia mediante modelos de corrección de error {[}Documento de Trabajo 26/01{]}},
  pdfauthor={Sergio A. Barona-Montoya},
  hidelinks,
  pdfcreator={LaTeX via pandoc}}

\title{\textbf{Transmisión de precios y ajuste asimétrico al equilibrio:
evidencia para las principales ciudades de Colombia mediante modelos de
corrección de error {[}Documento de Trabajo 26/01{]}}}
\author{Sergio A. Barona-Montoya\footnote{Department of Economics and
  Finance, Pontificia Universidad Javeriana, Cali, Colombia. ORCID
  0000-0001-8390-6673}}
\date{}

\begin{document}
\maketitle
\begin{abstract}
{[}Abstract text to be added{]}
\end{abstract}

\section{Introducción}\label{introducciuxf3n}

\section{Datos}\label{datos}

\subsection{Fuentes de datos: precios minoristas
(DANE--IPC)}\label{fuentes-de-datos-precios-minoristas-daneipc}

La primera fuente de información corresponde a los precios minoristas
reportados por el Departamento Administrativo Nacional de Estadística
(DANE), utilizados como insumo para el cálculo del Índice de Precios al
Consumidor (IPC) (DANE, 2024). El periodo de análisis se extiende desde
enero de 1999 hasta marzo de 2018 y cubre las trece principales ciudades
del país: Bogotá D.C., Medellín A.M., Cali A.M., Barranquilla A.M.,
Bucaramanga A.M., Manizales A.M., Pereira A.M., Cúcuta A.M., Pasto,
Ibagué, Montería, Cartagena y Villavicencio.

Una característica central de esta base de datos es su estructura de
clasificación, la cual se fundamenta en la canasta de seguimiento del
IPC 2008. Dicha estructura organiza la información de precios a través
de distintos niveles jerárquicos de agregación, lo que permite un
análisis flexible y consistente de la dinámica de precios a diferentes
escalas. En particular, la clasificación distingue entre cinco niveles:
grupo, subgrupo, clase, gasto básico y artículo.

\begin{longtable}[]{@{}
  >{\raggedright\arraybackslash}p{(\columnwidth - 14\tabcolsep) * \real{0.0617}}
  >{\raggedright\arraybackslash}p{(\columnwidth - 14\tabcolsep) * \real{0.0864}}
  >{\raggedright\arraybackslash}p{(\columnwidth - 14\tabcolsep) * \real{0.1728}}
  >{\raggedright\arraybackslash}p{(\columnwidth - 14\tabcolsep) * \real{0.1975}}
  >{\raggedright\arraybackslash}p{(\columnwidth - 14\tabcolsep) * \real{0.1975}}
  >{\raggedright\arraybackslash}p{(\columnwidth - 14\tabcolsep) * \real{0.0988}}
  >{\raggedright\arraybackslash}p{(\columnwidth - 14\tabcolsep) * \real{0.0988}}
  >{\raggedright\arraybackslash}p{(\columnwidth - 14\tabcolsep) * \real{0.0864}}@{}}
\caption{First rows of the IPC retail price dataset
(1999--2018)}\tabularnewline
\toprule\noalign{}
\begin{minipage}[b]{\linewidth}\raggedright
ano
\end{minipage} & \begin{minipage}[b]{\linewidth}\raggedright
ciudad
\end{minipage} & \begin{minipage}[b]{\linewidth}\raggedright
nombre\_ciudad
\end{minipage} & \begin{minipage}[b]{\linewidth}\raggedright
codigo\_articulo
\end{minipage} & \begin{minipage}[b]{\linewidth}\raggedright
articulo
\end{minipage} & \begin{minipage}[b]{\linewidth}\raggedright
unidad
\end{minipage} & \begin{minipage}[b]{\linewidth}\raggedright
mes
\end{minipage} & \begin{minipage}[b]{\linewidth}\raggedright
precio
\end{minipage} \\
\midrule\noalign{}
\endfirsthead
\toprule\noalign{}
\begin{minipage}[b]{\linewidth}\raggedright
ano
\end{minipage} & \begin{minipage}[b]{\linewidth}\raggedright
ciudad
\end{minipage} & \begin{minipage}[b]{\linewidth}\raggedright
nombre\_ciudad
\end{minipage} & \begin{minipage}[b]{\linewidth}\raggedright
codigo\_articulo
\end{minipage} & \begin{minipage}[b]{\linewidth}\raggedright
articulo
\end{minipage} & \begin{minipage}[b]{\linewidth}\raggedright
unidad
\end{minipage} & \begin{minipage}[b]{\linewidth}\raggedright
mes
\end{minipage} & \begin{minipage}[b]{\linewidth}\raggedright
precio
\end{minipage} \\
\midrule\noalign{}
\endhead
\bottomrule\noalign{}
\endlastfoot
1999 & 05 & MEDELLÍN & 1110101 & ARROZ PARA SECO & 500Grs. & enero &
588.81 \\
1999 & 05 & MEDELLÍN & 1110101 & ARROZ PARA SECO & 500Grs. & febrero &
588.37 \\
1999 & 05 & MEDELLÍN & 1110101 & ARROZ PARA SECO & 500Grs. & marzo &
587.72 \\
1999 & 05 & MEDELLÍN & 1110101 & ARROZ PARA SECO & 500Grs. & abril &
587.71 \\
1999 & 05 & MEDELLÍN & 1110101 & ARROZ PARA SECO & 500Grs. & mayo &
590.53 \\
1999 & 05 & MEDELLÍN & 1110101 & ARROZ PARA SECO & 500Grs. & junio &
592.17 \\
\end{longtable}

El nivel de \textbf{grupo} corresponde a la agregación más amplia de
bienes y servicios, y agrupa conjuntos homogéneos de consumo. Cada grupo
se subdivide en \textbf{subgrupos}, los cuales capturan categorías más
específicas dentro de cada conjunto. A su vez, los subgrupos se
desagregan en \textbf{clases}, que representan una segmentación aún más
detallada del consumo. El nivel de \textbf{gasto básico} constituye la
unidad operativa de seguimiento dentro del IPC y agrupa productos con
características similares y patrones de consumo comparables. Finalmente,
el nivel de \textbf{artículo} corresponde al producto específico cuyo
precio es observado y reportado de manera directa.

Esta estructura jerárquica garantiza coherencia entre los distintos
niveles de agregación y permite analizar la transmisión de precios desde
desagregaciones finas hasta agregados más amplios, lo cual resulta
particularmente relevante para la estimación de modelos de corrección de
error y modelos de corrección de error asimétricos a nivel urbano.

\subsection{Fuentes de datos: precios mayoristas (SIPSA -
DANE)}\label{fuentes-de-datos-precios-mayoristas-sipsa---dane}

La información sobre los precios mayoristas corresponde a los datos del
Sistema de Información de Precios y Abastecimiento del Sector
Agropecuario (SIPSA), publicados por el DANE. El SIPSA no sólo informa,
con frecuencia diaria, sobre los precios mayoristas de los productos
agroalimentarios que se comercializan en el país; sino que, además,
proporciona información, con frecuencia quincenal, sobre el nivel de
abastecimiento de los alimentos en las ciudades. En este estudio se
utilizarán datos de SIPSA para las tres principales ciudades de Colombia
(Cali, Bogotá y Medellín) durante el período 2013:1 -- 2024:1, con
frecuencia mensual.

A continuación, se presenta la estructura de los datos:

\begin{longtable}[]{@{}
  >{\raggedright\arraybackslash}p{(\columnwidth - 12\tabcolsep) * \real{0.1068}}
  >{\raggedright\arraybackslash}p{(\columnwidth - 12\tabcolsep) * \real{0.0680}}
  >{\raggedright\arraybackslash}p{(\columnwidth - 12\tabcolsep) * \real{0.2718}}
  >{\raggedright\arraybackslash}p{(\columnwidth - 12\tabcolsep) * \real{0.3495}}
  >{\raggedleft\arraybackslash}p{(\columnwidth - 12\tabcolsep) * \real{0.0971}}
  >{\raggedleft\arraybackslash}p{(\columnwidth - 12\tabcolsep) * \real{0.0583}}
  >{\raggedleft\arraybackslash}p{(\columnwidth - 12\tabcolsep) * \real{0.0485}}@{}}
\caption{First rows of the SIPSA wholesale price dataset
(2013--2018)}\tabularnewline
\toprule\noalign{}
\begin{minipage}[b]{\linewidth}\raggedright
Fecha
\end{minipage} & \begin{minipage}[b]{\linewidth}\raggedright
Grupo
\end{minipage} & \begin{minipage}[b]{\linewidth}\raggedright
Alimento
\end{minipage} & \begin{minipage}[b]{\linewidth}\raggedright
Mercado
\end{minipage} & \begin{minipage}[b]{\linewidth}\raggedleft
Precio\_kg
\end{minipage} & \begin{minipage}[b]{\linewidth}\raggedleft
Month
\end{minipage} & \begin{minipage}[b]{\linewidth}\raggedleft
Year
\end{minipage} \\
\midrule\noalign{}
\endfirsthead
\toprule\noalign{}
\begin{minipage}[b]{\linewidth}\raggedright
Fecha
\end{minipage} & \begin{minipage}[b]{\linewidth}\raggedright
Grupo
\end{minipage} & \begin{minipage}[b]{\linewidth}\raggedright
Alimento
\end{minipage} & \begin{minipage}[b]{\linewidth}\raggedright
Mercado
\end{minipage} & \begin{minipage}[b]{\linewidth}\raggedleft
Precio\_kg
\end{minipage} & \begin{minipage}[b]{\linewidth}\raggedleft
Month
\end{minipage} & \begin{minipage}[b]{\linewidth}\raggedleft
Year
\end{minipage} \\
\midrule\noalign{}
\endhead
\bottomrule\noalign{}
\endlastfoot
2013-01-01 & CARNES & Alas de pollo con costillar & Barranquilla,
Barranquillita & 3073 & 1 & 2013 \\
2013-01-01 & CARNES & Alas de pollo con costillar & Barranquilla,
Granabastos & 3067 & 1 & 2013 \\
2013-01-01 & CARNES & Alas de pollo con costillar & Bogotá, D.C.,
Frigorífico Ble Ltda. & 2967 & 1 & 2013 \\
2013-01-01 & CARNES & Alas de pollo con costillar & Bogotá, D.C.,
Frigorífico Guadalupe & 2560 & 1 & 2013 \\
2013-01-01 & CARNES & Alas de pollo con costillar & Cartagena, Bazurto &
3429 & 1 & 2013 \\
2013-01-01 & CARNES & Alas de pollo con costillar & Ibagué, Plaza La 21
& 4307 & 1 & 2013 \\
\end{longtable}

\section{Metodología}\label{metodologuxeda}

\subsection{Mapeo IPC--SIPSA}\label{mapeo-ipcsipsa}

La implementación de la metodología presupone un mapeo previo entre los
alimentos reportados por el DANE en la construcción del IPC y los
alimentos reportados por SIPSA (en adelante, \textbf{mapeo IPC--SIPSA}).
En términos operativos, este procedimiento establece correspondencias
entre ambas fuentes con el fin de asegurar comparabilidad conceptual y
estadística en las series de precios. En particular, para cada alimento
reportado en SIPSA se identifica un alimento equivalente dentro de la
canasta del IPC. A continuación, se presenta el mapeo IPC--SIPSA.

\begin{longtable}[]{@{}llll@{}}
\caption{Mapeo IPC-SIPSA}\tabularnewline
\toprule\noalign{}
codigo\_tcac & retail & mapeo\_sipsa & sipsa \\
\midrule\noalign{}
\endfirsthead
\toprule\noalign{}
codigo\_tcac & retail & mapeo\_sipsa & sipsa \\
\midrule\noalign{}
\endhead
\bottomrule\noalign{}
\endlastfoot
26 & FECULA DE MAÍZ & 26 & Fécula de maíz \\
A010 & ARROZ PARA SECO & A010 & Arroz blanco importado \\
A010 & ARROZ PARA SECO & A010 & Arroz de primera \\
A010 & ARROZ PARA SECO & A010 & Arroz de segunda \\
A010 & ARROZ PARA SECO & A010 & Arroz excelso \\
A012 & AVENA HOJUELAS & A012 & Avena en hojuelas \\
\end{longtable}

Es importante destacar que el mapeo no sigue una relación uno a uno,
sino una relación \textbf{n a 1} (muchos a uno). En consecuencia, un
mismo alimento definido en la base del IPC puede agrupar varios
alimentos equivalentes reportados en SIPSA. (Por ejemplo, el ítem
\emph{arroz para seco} en el IPC puede corresponder, en SIPSA, a
categorías como \emph{arroz de primera}, \emph{arroz de segunda} y
\emph{arroz excelso}). Este tipo de correspondencia refleja diferencias
en el nivel de desagregación y en los criterios de clasificación entre
las dos fuentes, y constituye un elemento central para la correcta
construcción de series comparables en el análisis de transmisión de
precios.

La siguiente tabla presenta el mapeo final utilizado en el análisis
posterior:

\begin{longtable}[]{@{}
  >{\raggedright\arraybackslash}p{(\columnwidth - 8\tabcolsep) * \real{0.2099}}
  >{\raggedright\arraybackslash}p{(\columnwidth - 8\tabcolsep) * \real{0.4074}}
  >{\raggedright\arraybackslash}p{(\columnwidth - 8\tabcolsep) * \real{0.0988}}
  >{\raggedleft\arraybackslash}p{(\columnwidth - 8\tabcolsep) * \real{0.1481}}
  >{\raggedleft\arraybackslash}p{(\columnwidth - 8\tabcolsep) * \real{0.1358}}@{}}
\caption{Mapeo IPC-SIPSA: Alimentos incluidos en el análisis (Cali,
Bogotá, Medellín)}\tabularnewline
\toprule\noalign{}
\begin{minipage}[b]{\linewidth}\raggedright
articulo\_ipc
\end{minipage} & \begin{minipage}[b]{\linewidth}\raggedright
alimento\_sipsa
\end{minipage} & \begin{minipage}[b]{\linewidth}\raggedright
cod\_mun
\end{minipage} & \begin{minipage}[b]{\linewidth}\raggedleft
months\_both
\end{minipage} & \begin{minipage}[b]{\linewidth}\raggedleft
months\_any
\end{minipage} \\
\midrule\noalign{}
\endfirsthead
\toprule\noalign{}
\begin{minipage}[b]{\linewidth}\raggedright
articulo\_ipc
\end{minipage} & \begin{minipage}[b]{\linewidth}\raggedright
alimento\_sipsa
\end{minipage} & \begin{minipage}[b]{\linewidth}\raggedright
cod\_mun
\end{minipage} & \begin{minipage}[b]{\linewidth}\raggedleft
months\_both
\end{minipage} & \begin{minipage}[b]{\linewidth}\raggedleft
months\_any
\end{minipage} \\
\midrule\noalign{}
\endhead
\bottomrule\noalign{}
\endlastfoot
ARROZ PARA SECO & Arroz de primera & 05001 & 63 & 63 \\
ARROZ PARA SECO & Arroz de primera & 11001 & 63 & 63 \\
ARROZ PARA SECO & Arroz de primera & 76001 & 63 & 63 \\
CEBOLLA CABEZONA & Cebolla cabezona blanca & 05001 & 63 & 63 \\
CEBOLLA CABEZONA & Cebolla cabezona blanca & 11001 & 63 & 63 \\
CEBOLLA CABEZONA & Cebolla cabezona blanca bogotana & 76001 & 63 & 63 \\
PAPA & Papa R-12 negra & 11001 & 63 & 63 \\
PAPA & Papa capira & 05001 & 63 & 63 \\
PAPA & Papa capira & 76001 & 63 & 63 \\
PLÁTANO & Plátano hartón verde & 05001 & 63 & 63 \\
PLÁTANO & Plátano hartón verde & 11001 & 63 & 63 \\
PLÁTANO & Plátano hartón verde & 76001 & 63 & 63 \\
TOMATE & Tomate larga vida & 05001 & 63 & 63 \\
TOMATE & Tomate larga vida & 11001 & 63 & 63 \\
TOMATE & Tomate larga vida & 76001 & 63 & 63 \\
YUCA & Yuca ICA & 05001 & 63 & 63 \\
YUCA & Yuca ICA & 76001 & 63 & 63 \\
YUCA & Yuca llanera & 11001 & 63 & 63 \\
ZANAHORIA & Zanahoria & 11001 & 63 & 63 \\
ZANAHORIA & Zanahoria bogotana & 76001 & 63 & 63 \\
ZANAHORIA & Zanahoria larga vida & 05001 & 63 & 63 \\
\end{longtable}

\subsection{Modelo de corrección de
error}\label{modelo-de-correcciuxf3n-de-error}

A partir de pruebas de Dickey--Fuller aumentadas (ADF), se examina la
estacionariedad de las series temporales correspondientes a los precios
de los alimentos analizados. La existencia de relaciones de
cointegración entre los precios minoristas y mayoristas se evalúa
mediante la prueba de Engle y Granger (1987).

Como señala Enders (2014), una característica fundamental de las
variables cointegradas es que su trayectoria temporal está determinada
por las desviaciones respecto del equilibrio de largo plazo. En
consecuencia, las dinámicas de corto plazo deben estar condicionadas por
dichas desviaciones. El modelo dinámico que permite capturar
simultáneamente la relación de largo plazo y los ajustes de corto plazo
es el modelo de corrección de error (\emph{Error Correction Model},
ECM).

\subsubsection{Modelo de corrección de error estándar
(ECM)}\label{modelo-de-correcciuxf3n-de-error-estuxe1ndar-ecm}

Sea \(\ln(P_{it}^{min})\) el logaritmo natural del precio minorista y
sea \(\ln(P_{it}^{may})\) el logaritmo natural del precio mayorista del
alimento \(i\) en el período \(t\). La relación de equilibrio de largo
plazo entre ambos precios viene dada por:

\[
\ln(P_{it}^{min}) = \alpha_i + \beta_i \ln(P_{it}^{may}) + e_{it}.
\]

El término de corrección de error corresponde al residuo rezagado de la
relación de cointegración:

\[
e_{i,t-1} = \ln(P_{i,t-1}^{min}) - \alpha_i - \beta_i \ln(P_{i,t-1}^{may}).
\]

Este término captura el desequilibrio existente en el período \(t-1\).
Si \(e_{i,t-1} > 0\), el precio minorista se encuentra por encima del
nivel consistente con el equilibrio de largo plazo, dado el precio
mayorista. En este contexto, la dinámica de corto plazo se modela
mediante la siguiente especificación ECM:

\[
\Delta \ln(P_{it}^{min}) =
c_{i0}
+ \sum_{p=1}^{P} \beta_{ip} \, \Delta \ln(P_{i,t-p}^{min})
+ \sum_{q=1}^{Q} \gamma_{iq} \, \Delta \ln(P_{i,t-q}^{may})
+ \theta_i \, e_{i,t-1}
+ u_{it}.
\]

En esta ecuación, los coeficientes \(\beta_{ip}\) capturan la dinámica
autorregresiva de corto plazo del precio minorista, mientras que los
coeficientes \(\gamma_{iq}\) reflejan el impacto de corto plazo de
variaciones en el precio mayorista. El parámetro \(\theta_i\) mide la
velocidad de ajuste hacia el equilibrio de largo plazo. Se espera que
\(\theta_i < 0\), de modo que desviaciones positivas del equilibrio sean
corregidas mediante reducciones en el crecimiento del precio minorista.

\subsubsection{Modelo de corrección de error asimétrico
(A-ECM)}\label{modelo-de-correcciuxf3n-de-error-asimuxe9trico-a-ecm}

Siguiendo estudios previos sobre transmisión asimétrica de precios (por
ejemplo, Chesnes, 2010), se implementa un modelo de corrección de error
que permite capturar posibles asimetrías tanto en la velocidad como en
el patrón de ajuste ante aumentos y reducciones en los precios. La
especificación del modelo de corrección de error asimétrico (A-ECM)
adopta la siguiente forma:

\[
\begin{aligned}
\Delta \ln(P_t^{min}) =\;&
\sum_{i=0}^{L_1^+} \beta_{1i}^+ \, \Delta^+ \ln(P_{t-i}^{may})
+ \sum_{i=0}^{L_1^-} \beta_{1i}^- \, \Delta^- \ln(P_{t-i}^{may}) \\
&+ \sum_{i=0}^{L_2^+} \beta_{2i}^+ \, \Delta^+ \ln(P_{t-i}^{min})
+ \sum_{i=0}^{L_2^-} \beta_{2i}^- \, \Delta^- \ln(P_{t-i}^{min}) \\
&+ \beta_3^+ \, e_{t-1}^+
+ \beta_3^- \, e_{t-1}^-
+ u_t .
\end{aligned}
\]

En esta expresión, \(\Delta^+\) y \(\Delta^-\) representan las
variaciones positivas y negativas de las variables respectivas. El
término de corrección de error \(e_{t-1}\) captura la relación de
equilibrio de largo plazo entre el precio minorista y el precio
mayorista y se descompone en sus componentes positivo y negativo,
\(e_{t-1}^+\) y \(e_{t-1}^-\).

Se espera que ambos coeficientes \(\beta_3^+\) y \(\beta_3^-\) sean
negativos. En particular, si el precio minorista se encuentra por encima
del equilibrio de largo plazo (\(e_{t-1} > 0\)), el ajuste debería
materializarse a través de una reducción en el crecimiento del precio
minorista; de manera análoga, si el precio minorista se sitúa por debajo
del equilibrio (\(e_{t-1} < 0\)), el ajuste debería reflejarse en un
aumento de dicho crecimiento.

Siguiendo la metodología en dos etapas propuesta por Engle y Granger
(1987), la relación de largo plazo se estima a partir de la siguiente
ecuación:

\[
\ln(P_{t-1}^{min}) = \alpha_0 + \beta_1 \ln(P_{t-1}^{may}) + e_{t-1}.
\]

Los residuales estimados de esta ecuación se incorporan posteriormente
en la especificación A-ECM como términos de corrección de error,
permitiendo evaluar empíricamente la presencia de ajustes asimétricos en
la transmisión de precios.

\section{Resultados}\label{resultados}

\subsection{Análisis de series de
tiempo}\label{anuxe1lisis-de-series-de-tiempo}

\subsubsection{Comportamiento de los precios minoristas y precios
mayoristas}\label{comportamiento-de-los-precios-minoristas-y-precios-mayoristas}

\begin{figure}[htbp]
\centering
\includegraphics{output/ts-output/log-levels-plots/261225_log_retail_vs_wholesale_7x3.png}
\caption{Log retail (IPC) and wholesale (SIPSA) prices by food and city}
\end{figure}

\begin{figure}[htbp]
\centering
\includegraphics{output/ts-output/seasonal-plots/261225_seasonal_retail_vs_wholesale_7x3_x13.png}
\caption{Log retail (IPC) and wholesale (SIPSA) prices by food and city}
\end{figure}

\subsubsection{Pruebas de raíz
unitaria}\label{pruebas-de-rauxedz-unitaria}

\begin{longtable}[]{@{}
  >{\raggedright\arraybackslash}p{(\columnwidth - 14\tabcolsep) * \real{0.0732}}
  >{\raggedright\arraybackslash}p{(\columnwidth - 14\tabcolsep) * \real{0.2073}}
  >{\raggedright\arraybackslash}p{(\columnwidth - 14\tabcolsep) * \real{0.2195}}
  >{\centering\arraybackslash}p{(\columnwidth - 14\tabcolsep) * \real{0.1829}}
  >{\centering\arraybackslash}p{(\columnwidth - 14\tabcolsep) * \real{0.0976}}
  >{\raggedright\arraybackslash}p{(\columnwidth - 14\tabcolsep) * \real{0.0854}}
  >{\centering\arraybackslash}p{(\columnwidth - 14\tabcolsep) * \real{0.0976}}
  >{\raggedright\arraybackslash}p{(\columnwidth - 14\tabcolsep) * \real{0.0366}}@{}}
\caption{Augmented Dickey Fuller tests on seasonally adjusted log retail
prices (IPC). Notes: X 13 seasonal adjustment is applied to log price
series; reported critical values correspond to the Dickey Fuller
distribution. Significance: *** p\textless0.01, ** p\textless0.05, *
p\textless0.10.}\tabularnewline
\toprule\noalign{}
\begin{minipage}[b]{\linewidth}\raggedright
City
\end{minipage} & \begin{minipage}[b]{\linewidth}\raggedright
Item
\end{minipage} & \begin{minipage}[b]{\linewidth}\raggedright
Specification
\end{minipage} & \begin{minipage}[b]{\linewidth}\centering
ADF statistic
\end{minipage} & \begin{minipage}[b]{\linewidth}\centering
CV 1\%
\end{minipage} & \begin{minipage}[b]{\linewidth}\raggedright
CV 5\%
\end{minipage} & \begin{minipage}[b]{\linewidth}\centering
CV 10\%
\end{minipage} & \begin{minipage}[b]{\linewidth}\raggedright
N
\end{minipage} \\
\midrule\noalign{}
\endfirsthead
\toprule\noalign{}
\begin{minipage}[b]{\linewidth}\raggedright
City
\end{minipage} & \begin{minipage}[b]{\linewidth}\raggedright
Item
\end{minipage} & \begin{minipage}[b]{\linewidth}\raggedright
Specification
\end{minipage} & \begin{minipage}[b]{\linewidth}\centering
ADF statistic
\end{minipage} & \begin{minipage}[b]{\linewidth}\centering
CV 1\%
\end{minipage} & \begin{minipage}[b]{\linewidth}\raggedright
CV 5\%
\end{minipage} & \begin{minipage}[b]{\linewidth}\centering
CV 10\%
\end{minipage} & \begin{minipage}[b]{\linewidth}\raggedright
N
\end{minipage} \\
\midrule\noalign{}
\endhead
\bottomrule\noalign{}
\endlastfoot
05001 & ARROZ PARA SECO & Intercept & -1.823 & -3.510 & -2.890 & -2.580
& 63 \\
05001 & ARROZ PARA SECO & Intercept + trend & -1.141 & -4.040 & -3.450 &
-3.150 & 63 \\
05001 & CEBOLLA CABEZONA & Intercept & -3.502** & -3.510 & -2.890 &
-2.580 & 63 \\
05001 & CEBOLLA CABEZONA & Intercept + trend & -4.328*** & -4.040 &
-3.450 & -3.150 & 63 \\
05001 & PAPA & Intercept & -2.657* & -3.510 & -2.890 & -2.580 & 63 \\
05001 & PAPA & Intercept + trend & -2.732 & -4.040 & -3.450 & -3.150 &
63 \\
05001 & PLÁTANO & Intercept & -1.229 & -3.510 & -2.890 & -2.580 & 63 \\
05001 & PLÁTANO & Intercept + trend & -1.667 & -4.040 & -3.450 & -3.150
& 63 \\
05001 & TOMATE & Intercept & -3.859*** & -3.510 & -2.890 & -2.580 &
63 \\
05001 & TOMATE & Intercept + trend & -4.602*** & -4.040 & -3.450 &
-3.150 & 63 \\
05001 & YUCA & Intercept & -1.425 & -3.510 & -2.890 & -2.580 & 63 \\
05001 & YUCA & Intercept + trend & -1.906 & -4.040 & -3.450 & -3.150 &
63 \\
05001 & ZANAHORIA & Intercept & -2.374 & -3.510 & -2.890 & -2.580 &
63 \\
05001 & ZANAHORIA & Intercept + trend & -3.842** & -4.040 & -3.450 &
-3.150 & 63 \\
11001 & ARROZ PARA SECO & Intercept & -1.755 & -3.510 & -2.890 & -2.580
& 63 \\
11001 & ARROZ PARA SECO & Intercept + trend & -0.998 & -4.040 & -3.450 &
-3.150 & 63 \\
11001 & CEBOLLA CABEZONA & Intercept & -3.270** & -3.510 & -2.890 &
-2.580 & 63 \\
11001 & CEBOLLA CABEZONA & Intercept + trend & -3.749** & -4.040 &
-3.450 & -3.150 & 63 \\
11001 & PAPA & Intercept & -3.224** & -3.510 & -2.890 & -2.580 & 63 \\
11001 & PAPA & Intercept + trend & -3.171* & -4.040 & -3.450 & -3.150 &
63 \\
11001 & PLÁTANO & Intercept & -1.990 & -3.510 & -2.890 & -2.580 & 63 \\
11001 & PLÁTANO & Intercept + trend & -2.266 & -4.040 & -3.450 & -3.150
& 63 \\
11001 & TOMATE & Intercept & -3.684*** & -3.510 & -2.890 & -2.580 &
63 \\
11001 & TOMATE & Intercept + trend & -4.022** & -4.040 & -3.450 & -3.150
& 63 \\
11001 & YUCA & Intercept & -0.822 & -3.510 & -2.890 & -2.580 & 63 \\
11001 & YUCA & Intercept + trend & -0.786 & -4.040 & -3.450 & -3.150 &
63 \\
11001 & ZANAHORIA & Intercept & -2.289 & -3.510 & -2.890 & -2.580 &
63 \\
11001 & ZANAHORIA & Intercept + trend & -3.399* & -4.040 & -3.450 &
-3.150 & 63 \\
76001 & ARROZ PARA SECO & Intercept & -1.883 & -3.510 & -2.890 & -2.580
& 63 \\
76001 & ARROZ PARA SECO & Intercept + trend & -1.196 & -4.040 & -3.450 &
-3.150 & 63 \\
76001 & CEBOLLA CABEZONA & Intercept & -3.077** & -3.510 & -2.890 &
-2.580 & 63 \\
76001 & CEBOLLA CABEZONA & Intercept + trend & -3.550** & -4.040 &
-3.450 & -3.150 & 63 \\
76001 & PAPA & Intercept & -2.534 & -3.510 & -2.890 & -2.580 & 63 \\
76001 & PAPA & Intercept + trend & -2.630 & -4.040 & -3.450 & -3.150 &
63 \\
76001 & PLÁTANO & Intercept & -1.547 & -3.510 & -2.890 & -2.580 & 63 \\
76001 & PLÁTANO & Intercept + trend & -1.757 & -4.040 & -3.450 & -3.150
& 63 \\
76001 & TOMATE & Intercept & -3.541*** & -3.510 & -2.890 & -2.580 &
63 \\
76001 & TOMATE & Intercept + trend & -3.929** & -4.040 & -3.450 & -3.150
& 63 \\
76001 & YUCA & Intercept & -1.119 & -3.510 & -2.890 & -2.580 & 63 \\
76001 & YUCA & Intercept + trend & -1.322 & -4.040 & -3.450 & -3.150 &
63 \\
76001 & ZANAHORIA & Intercept & -2.632* & -3.510 & -2.890 & -2.580 &
63 \\
76001 & ZANAHORIA & Intercept + trend & -4.054*** & -4.040 & -3.450 &
-3.150 & 63 \\
\end{longtable}

\begin{longtable}[]{@{}
  >{\raggedright\arraybackslash}p{(\columnwidth - 14\tabcolsep) * \real{0.0612}}
  >{\raggedright\arraybackslash}p{(\columnwidth - 14\tabcolsep) * \real{0.3367}}
  >{\raggedright\arraybackslash}p{(\columnwidth - 14\tabcolsep) * \real{0.1837}}
  >{\centering\arraybackslash}p{(\columnwidth - 14\tabcolsep) * \real{0.1531}}
  >{\centering\arraybackslash}p{(\columnwidth - 14\tabcolsep) * \real{0.0816}}
  >{\raggedright\arraybackslash}p{(\columnwidth - 14\tabcolsep) * \real{0.0714}}
  >{\centering\arraybackslash}p{(\columnwidth - 14\tabcolsep) * \real{0.0816}}
  >{\raggedright\arraybackslash}p{(\columnwidth - 14\tabcolsep) * \real{0.0306}}@{}}
\caption{Augmented Dickey Fuller tests on seasonally adjusted log
wholesale prices (SIPSA). Notes: X 13 seasonal adjustment is applied to
log price series; reported critical values correspond to the Dickey
Fuller distribution. Significance: *** p\textless0.01, **
p\textless0.05, * p\textless0.10.}\tabularnewline
\toprule\noalign{}
\begin{minipage}[b]{\linewidth}\raggedright
City
\end{minipage} & \begin{minipage}[b]{\linewidth}\raggedright
Item
\end{minipage} & \begin{minipage}[b]{\linewidth}\raggedright
Specification
\end{minipage} & \begin{minipage}[b]{\linewidth}\centering
ADF statistic
\end{minipage} & \begin{minipage}[b]{\linewidth}\centering
CV 1\%
\end{minipage} & \begin{minipage}[b]{\linewidth}\raggedright
CV 5\%
\end{minipage} & \begin{minipage}[b]{\linewidth}\centering
CV 10\%
\end{minipage} & \begin{minipage}[b]{\linewidth}\raggedright
N
\end{minipage} \\
\midrule\noalign{}
\endfirsthead
\toprule\noalign{}
\begin{minipage}[b]{\linewidth}\raggedright
City
\end{minipage} & \begin{minipage}[b]{\linewidth}\raggedright
Item
\end{minipage} & \begin{minipage}[b]{\linewidth}\raggedright
Specification
\end{minipage} & \begin{minipage}[b]{\linewidth}\centering
ADF statistic
\end{minipage} & \begin{minipage}[b]{\linewidth}\centering
CV 1\%
\end{minipage} & \begin{minipage}[b]{\linewidth}\raggedright
CV 5\%
\end{minipage} & \begin{minipage}[b]{\linewidth}\centering
CV 10\%
\end{minipage} & \begin{minipage}[b]{\linewidth}\raggedright
N
\end{minipage} \\
\midrule\noalign{}
\endhead
\bottomrule\noalign{}
\endlastfoot
05001 & Arroz de primera & Intercept & -1.461 & -3.510 & -2.890 & -2.580
& 63 \\
05001 & Arroz de primera & Intercept + trend & -0.984 & -4.040 & -3.450
& -3.150 & 63 \\
05001 & Cebolla cabezona blanca & Intercept & -2.888* & -3.510 & -2.890
& -2.580 & 63 \\
05001 & Cebolla cabezona blanca & Intercept + trend & -2.666 & -4.040 &
-3.450 & -3.150 & 63 \\
05001 & Papa capira & Intercept & -2.407 & -3.510 & -2.890 & -2.580 &
63 \\
05001 & Papa capira & Intercept + trend & -2.357 & -4.040 & -3.450 &
-3.150 & 63 \\
05001 & Plátano hartón verde & Intercept & -1.465 & -3.510 & -2.890 &
-2.580 & 63 \\
05001 & Plátano hartón verde & Intercept + trend & -1.437 & -4.040 &
-3.450 & -3.150 & 63 \\
05001 & Tomate larga vida & Intercept & -1.229 & -3.510 & -2.890 &
-2.580 & 63 \\
05001 & Tomate larga vida & Intercept + trend & -2.299 & -4.040 & -3.450
& -3.150 & 63 \\
05001 & Yuca ICA & Intercept & -1.425 & -3.510 & -2.890 & -2.580 & 63 \\
05001 & Yuca ICA & Intercept + trend & -1.516 & -4.040 & -3.450 & -3.150
& 63 \\
05001 & Zanahoria larga vida & Intercept & -2.182 & -3.510 & -2.890 &
-2.580 & 63 \\
05001 & Zanahoria larga vida & Intercept + trend & -2.227 & -4.040 &
-3.450 & -3.150 & 63 \\
11001 & Arroz de primera & Intercept & -1.350 & -3.510 & -2.890 & -2.580
& 63 \\
11001 & Arroz de primera & Intercept + trend & -1.010 & -4.040 & -3.450
& -3.150 & 63 \\
11001 & Cebolla cabezona blanca & Intercept & -2.875* & -3.510 & -2.890
& -2.580 & 63 \\
11001 & Cebolla cabezona blanca & Intercept + trend & -2.723 & -4.040 &
-3.450 & -3.150 & 63 \\
11001 & Papa R-12 negra & Intercept & -2.300 & -3.510 & -2.890 & -2.580
& 63 \\
11001 & Papa R-12 negra & Intercept + trend & -2.209 & -4.040 & -3.450 &
-3.150 & 63 \\
11001 & Plátano hartón verde & Intercept & -1.881 & -3.510 & -2.890 &
-2.580 & 63 \\
11001 & Plátano hartón verde & Intercept + trend & -1.980 & -4.040 &
-3.450 & -3.150 & 63 \\
11001 & Tomate larga vida & Intercept & -1.673 & -3.510 & -2.890 &
-2.580 & 63 \\
11001 & Tomate larga vida & Intercept + trend & -2.301 & -4.040 & -3.450
& -3.150 & 63 \\
11001 & Yuca llanera & Intercept & -1.588 & -3.510 & -2.890 & -2.580 &
63 \\
11001 & Yuca llanera & Intercept + trend & -1.550 & -4.040 & -3.450 &
-3.150 & 63 \\
11001 & Zanahoria & Intercept & -2.036 & -3.510 & -2.890 & -2.580 &
63 \\
11001 & Zanahoria & Intercept + trend & -2.157 & -4.040 & -3.450 &
-3.150 & 63 \\
76001 & Arroz de primera & Intercept & -1.361 & -3.510 & -2.890 & -2.580
& 63 \\
76001 & Arroz de primera & Intercept + trend & -0.573 & -4.040 & -3.450
& -3.150 & 63 \\
76001 & Cebolla cabezona blanca bogotana & Intercept & -2.792* & -3.510
& -2.890 & -2.580 & 63 \\
76001 & Cebolla cabezona blanca bogotana & Intercept + trend & -2.660 &
-4.040 & -3.450 & -3.150 & 63 \\
76001 & Papa capira & Intercept & -2.583* & -3.510 & -2.890 & -2.580 &
63 \\
76001 & Papa capira & Intercept + trend & -2.487 & -4.040 & -3.450 &
-3.150 & 63 \\
76001 & Plátano hartón verde & Intercept & -1.472 & -3.510 & -2.890 &
-2.580 & 63 \\
76001 & Plátano hartón verde & Intercept + trend & -1.496 & -4.040 &
-3.450 & -3.150 & 63 \\
76001 & Tomate larga vida & Intercept & -1.310 & -3.510 & -2.890 &
-2.580 & 63 \\
76001 & Tomate larga vida & Intercept + trend & -2.336 & -4.040 & -3.450
& -3.150 & 63 \\
76001 & Yuca ICA & Intercept & -1.428 & -3.510 & -2.890 & -2.580 & 63 \\
76001 & Yuca ICA & Intercept + trend & -1.659 & -4.040 & -3.450 & -3.150
& 63 \\
76001 & Zanahoria bogotana & Intercept & -2.208 & -3.510 & -2.890 &
-2.580 & 63 \\
76001 & Zanahoria bogotana & Intercept + trend & -2.419 & -4.040 &
-3.450 & -3.150 & 63 \\
\end{longtable}

\subsection{Análisis de cointegración de
Engle-Granger}\label{anuxe1lisis-de-cointegraciuxf3n-de-engle-granger}

Las Tablas X e Y reportan los resultados de los test de cointegración de
Engle-Granger entre los precios minoristas (IPC) y mayoristas (SIPSA),
expresados en logaritmos. Puesto que el estadístico dependen de la
especificación, la longitud del rezago en la regresión ``auxiliar2 se
selecciona de acuerdo con criterios de información (AIC y BIC). La Tabla
X presenta los resultados cuando la selección del rezago óptimo se
realiza a partir de una regresión ADF con constante
(especificación''drift''). En contraste, la Tabla Y reporta los
resultados correspondientes a una regresión ADF que incluye tanto la
constante como la tendencia lineal. Finalmente, la inferencia final
sobre la cointegración es implementada usando la función ´coint.test´,
que considera los valores críticos y p-valores de MacKinnon.

\begingroup\fontsize{9}{11}\selectfont

\begin{longtable}[t]{lllclc}
\caption{\label{tab:coint-eg-q1-tables}Engle–Granger cointegration tests using seasonally adjusted log prices. Lag length selected by BIC from residual ADF regression with intercept.}\\
\toprule
City & Retail item (IPC) & Wholesale item (SIPSA) & Specification & EG statistic & p-value\\
\midrule
Bogotá & ARROZ PARA SECO & Arroz de primera & type 1 & -2.392 & 0.100\\
Bogotá & ARROZ PARA SECO & Arroz de primera & type 2 & -0.212 & 0.100\\
Bogotá & CEBOLLA CABEZONA & Cebolla cabezona blanca & type 1 & -2.400 & 0.100\\
Bogotá & CEBOLLA CABEZONA & Cebolla cabezona blanca & type 2 & 0.760 & 0.100\\
Bogotá & PAPA & Papa R-12 negra & type 1 & -1.977 & 0.100\\
\addlinespace
Bogotá & PAPA & Papa R-12 negra & type 2 & -0.070 & 0.100\\
Bogotá & PLÁTANO & Plátano hartón verde & type 1 & -2.307 & 0.100\\
Bogotá & PLÁTANO & Plátano hartón verde & type 2 & 0.626 & 0.100\\
Bogotá & TOMATE & Tomate larga vida & type 1 & -4.015** & 0.010\\
Bogotá & TOMATE & Tomate larga vida & type 2 & 0.042 & 0.100\\
\addlinespace
Bogotá & YUCA & Yuca llanera & type 1 & -1.512 & 0.100\\
Bogotá & YUCA & Yuca llanera & type 2 & -1.411 & 0.100\\
Bogotá & ZANAHORIA & Zanahoria & type 1 & -2.225 & 0.100\\
Bogotá & ZANAHORIA & Zanahoria & type 2 & 0.984 & 0.100\\
Cali & ARROZ PARA SECO & Arroz de primera & type 1 & -4.331** & 0.010\\
\addlinespace
Cali & ARROZ PARA SECO & Arroz de primera & type 2 & -0.141 & 0.100\\
Cali & CEBOLLA CABEZONA & Cebolla cabezona blanca bogotana & type 1 & -2.817* & 0.065\\
Cali & CEBOLLA CABEZONA & Cebolla cabezona blanca bogotana & type 2 & 0.725 & 0.100\\
Cali & PAPA & Papa capira & type 1 & -2.404 & 0.100\\
Cali & PAPA & Papa capira & type 2 & 0.355 & 0.100\\
\addlinespace
Cali & PLÁTANO & Plátano hartón verde & type 1 & -2.480 & 0.100\\
Cali & PLÁTANO & Plátano hartón verde & type 2 & 0.861 & 0.100\\
Cali & TOMATE & Tomate larga vida & type 1 & -4.810** & 0.010\\
Cali & TOMATE & Tomate larga vida & type 2 & -0.125 & 0.100\\
Cali & YUCA & Yuca ICA & type 1 & -1.624 & 0.100\\
\addlinespace
Cali & YUCA & Yuca ICA & type 2 & -0.479 & 0.100\\
Cali & ZANAHORIA & Zanahoria bogotana & type 1 & -2.276 & 0.100\\
Cali & ZANAHORIA & Zanahoria bogotana & type 2 & 1.047 & 0.100\\
Medellín & ARROZ PARA SECO & Arroz de primera & type 1 & -2.850* & 0.059\\
Medellín & ARROZ PARA SECO & Arroz de primera & type 2 & -0.133 & 0.100\\
\addlinespace
Medellín & CEBOLLA CABEZONA & Cebolla cabezona blanca & type 1 & -2.418 & 0.100\\
Medellín & CEBOLLA CABEZONA & Cebolla cabezona blanca & type 2 & 0.817 & 0.100\\
Medellín & PAPA & Papa capira & type 1 & -2.688* & 0.085\\
Medellín & PAPA & Papa capira & type 2 & 0.440 & 0.100\\
Medellín & PLÁTANO & Plátano hartón verde & type 1 & -1.666 & 0.100\\
\addlinespace
Medellín & PLÁTANO & Plátano hartón verde & type 2 & 1.178 & 0.100\\
Medellín & TOMATE & Tomate larga vida & type 1 & -4.018** & 0.010\\
Medellín & TOMATE & Tomate larga vida & type 2 & -0.063 & 0.100\\
Medellín & YUCA & Yuca ICA & type 1 & -2.773* & 0.071\\
Medellín & YUCA & Yuca ICA & type 2 & 0.151 & 0.100\\
\addlinespace
Medellín & ZANAHORIA & Zanahoria larga vida & type 1 & -2.006 & 0.100\\
Medellín & ZANAHORIA & Zanahoria larga vida & type 2 & 1.123 & 0.100\\
\bottomrule
\end{longtable}
\endgroup{}

\begingroup\fontsize{9}{11}\selectfont

\begin{longtable}[t]{lllclc}
\caption{\label{tab:coint-eg-q1-tables}Engle–Granger cointegration tests using seasonally adjusted log prices. Lag length selected by BIC from residual ADF regression with intercept and trend.}\\
\toprule
City & Retail item (IPC) & Wholesale item (SIPSA) & Specification & EG statistic & p-value\\
\midrule
Bogotá & ARROZ PARA SECO & Arroz de primera & type 1 & -2.392 & 0.100\\
Bogotá & ARROZ PARA SECO & Arroz de primera & type 2 & -0.212 & 0.100\\
Bogotá & CEBOLLA CABEZONA & Cebolla cabezona blanca & type 1 & -2.400 & 0.100\\
Bogotá & CEBOLLA CABEZONA & Cebolla cabezona blanca & type 2 & 0.760 & 0.100\\
Bogotá & PAPA & Papa R-12 negra & type 1 & -1.977 & 0.100\\
\addlinespace
Bogotá & PAPA & Papa R-12 negra & type 2 & -0.070 & 0.100\\
Bogotá & PLÁTANO & Plátano hartón verde & type 1 & -2.307 & 0.100\\
Bogotá & PLÁTANO & Plátano hartón verde & type 2 & 0.626 & 0.100\\
Bogotá & TOMATE & Tomate larga vida & type 1 & -4.015** & 0.010\\
Bogotá & TOMATE & Tomate larga vida & type 2 & 0.042 & 0.100\\
\addlinespace
Bogotá & YUCA & Yuca llanera & type 1 & -1.512 & 0.100\\
Bogotá & YUCA & Yuca llanera & type 2 & -1.411 & 0.100\\
Bogotá & ZANAHORIA & Zanahoria & type 1 & -2.225 & 0.100\\
Bogotá & ZANAHORIA & Zanahoria & type 2 & 0.984 & 0.100\\
Cali & ARROZ PARA SECO & Arroz de primera & type 1 & -4.331** & 0.010\\
\addlinespace
Cali & ARROZ PARA SECO & Arroz de primera & type 2 & -0.141 & 0.100\\
Cali & CEBOLLA CABEZONA & Cebolla cabezona blanca bogotana & type 1 & -2.817* & 0.065\\
Cali & CEBOLLA CABEZONA & Cebolla cabezona blanca bogotana & type 2 & 0.725 & 0.100\\
Cali & PAPA & Papa capira & type 1 & -2.404 & 0.100\\
Cali & PAPA & Papa capira & type 2 & 0.355 & 0.100\\
\addlinespace
Cali & PLÁTANO & Plátano hartón verde & type 1 & -2.480 & 0.100\\
Cali & PLÁTANO & Plátano hartón verde & type 2 & 0.861 & 0.100\\
Cali & TOMATE & Tomate larga vida & type 1 & -4.810** & 0.010\\
Cali & TOMATE & Tomate larga vida & type 2 & -0.125 & 0.100\\
Cali & YUCA & Yuca ICA & type 1 & -1.624 & 0.100\\
\addlinespace
Cali & YUCA & Yuca ICA & type 2 & -0.479 & 0.100\\
Cali & ZANAHORIA & Zanahoria bogotana & type 1 & -2.276 & 0.100\\
Cali & ZANAHORIA & Zanahoria bogotana & type 2 & 1.047 & 0.100\\
Medellín & ARROZ PARA SECO & Arroz de primera & type 1 & -2.850* & 0.059\\
Medellín & ARROZ PARA SECO & Arroz de primera & type 2 & -0.133 & 0.100\\
\addlinespace
Medellín & CEBOLLA CABEZONA & Cebolla cabezona blanca & type 1 & -2.418 & 0.100\\
Medellín & CEBOLLA CABEZONA & Cebolla cabezona blanca & type 2 & 0.817 & 0.100\\
Medellín & PAPA & Papa capira & type 1 & -2.688* & 0.085\\
Medellín & PAPA & Papa capira & type 2 & 0.440 & 0.100\\
Medellín & PLÁTANO & Plátano hartón verde & type 1 & -1.666 & 0.100\\
\addlinespace
Medellín & PLÁTANO & Plátano hartón verde & type 2 & 1.178 & 0.100\\
Medellín & TOMATE & Tomate larga vida & type 1 & -4.018** & 0.010\\
Medellín & TOMATE & Tomate larga vida & type 2 & -0.063 & 0.100\\
Medellín & YUCA & Yuca ICA & type 1 & -2.773* & 0.071\\
Medellín & YUCA & Yuca ICA & type 2 & 0.151 & 0.100\\
\addlinespace
Medellín & ZANAHORIA & Zanahoria larga vida & type 1 & -2.006 & 0.100\\
Medellín & ZANAHORIA & Zanahoria larga vida & type 2 & 1.123 & 0.100\\
\bottomrule
\end{longtable}
\endgroup{}

Adicionalmente, se consideran dos especificaciones estándar del test de
Engle-Granger. La especificación ``type 1'' corresponde a un modelo sin
tendencia; y la especificación ``type 2'', a un modelo con tendencia
lineal. En conjunción con los resultados de las pruebas de raíz
unitaria, los resultados de ambas tablas proporcionan evidencia en favor
de la presencia de cointegración para los siguientes casos: en Cali, el
arroz; en Medellín, el arroz, la papa y la yuca.

\subsubsection{Modelo de Corrección de Error simétrico
(ECM)}\label{modelo-de-correcciuxf3n-de-error-simuxe9trico-ecm}

A partir de las pruebas de raíz unitaria y el test de cointegración, el
análisis subsiguiente considera únicamente los siguientes cuatro
alimentos: (1) arroz, (2) papa, (3) plátano y (4) yuca. La siguiente
tabla presenta los resultados del modelo de corrección de error
simétrico (ECM). La longitud del rezago fue seleccionada a partir de
criterios de información (AIC y BIC).

\begin{table}[!h]
\centering
\caption{\label{tab:tab:longrun}Long-run relationship between log retail prices (IPC) and log wholesale prices (SIPSA) by city and pair. The table reports the slope estimate (b), standard error, and p-value from the long-run regression.}
\centering
\resizebox{\ifdim\width>\linewidth\linewidth\else\width\fi}{!}{
\fontsize{9}{11}\selectfont
\begin{tabular}[t]{lllll}
\toprule
city & articulo\_ipc & b & se & p-value\\
\midrule
\cellcolor{gray!10}{Bogotá} & \cellcolor{gray!10}{ARROZ PARA SECO} & \cellcolor{gray!10}{0.8597756} & \cellcolor{gray!10}{0.0614694} & \cellcolor{gray!10}{0.00e+00}\\
Cali & ARROZ PARA SECO & 0.8672186 & 0.0418960 & 0.00e+00\\
\cellcolor{gray!10}{Medellín} & \cellcolor{gray!10}{ARROZ PARA SECO} & \cellcolor{gray!10}{1.0053093} & \cellcolor{gray!10}{0.0689181} & \cellcolor{gray!10}{0.00e+00}\\
Bogotá & PAPA & 0.5335352 & 0.1115556 & 1.13e-05\\
\cellcolor{gray!10}{Cali} & \cellcolor{gray!10}{PAPA} & \cellcolor{gray!10}{0.7583602} & \cellcolor{gray!10}{0.1221081} & \cellcolor{gray!10}{1.00e-07}\\
\addlinespace
Medellín & PAPA & 1.0221967 & 0.1231101 & 0.00e+00\\
\cellcolor{gray!10}{Bogotá} & \cellcolor{gray!10}{PLÁTANO} & \cellcolor{gray!10}{0.7423630} & \cellcolor{gray!10}{0.0843069} & \cellcolor{gray!10}{0.00e+00}\\
Cali & PLÁTANO & 0.7291643 & 0.0630687 & 0.00e+00\\
\cellcolor{gray!10}{Medellín} & \cellcolor{gray!10}{PLÁTANO} & \cellcolor{gray!10}{0.7585014} & \cellcolor{gray!10}{0.0979122} & \cellcolor{gray!10}{0.00e+00}\\
Bogotá & YUCA & 0.4280256 & 0.0490270 & 0.00e+00\\
\addlinespace
\cellcolor{gray!10}{Cali} & \cellcolor{gray!10}{YUCA} & \cellcolor{gray!10}{0.5810043} & \cellcolor{gray!10}{0.0503080} & \cellcolor{gray!10}{0.00e+00}\\
Medellín & YUCA & 0.5375945 & 0.0446882 & 0.00e+00\\
\bottomrule
\end{tabular}}
\end{table}

\begin{table}[!h]
\centering
\caption{\label{tab:tab:ecm_cali}Error correction model (ECM) estimates for Cali. The ECM is estimated under the Enders restriction (same lag length for \(\Delta y\) and \(\Delta x\), no contemporaneous \(\Delta x_t\)).}
\centering
\resizebox{\ifdim\width>\linewidth\linewidth\else\width\fi}{!}{
\fontsize{9}{11}\selectfont
\begin{tabular}[t]{lllll}
\toprule
pair & term & b & se & p-value\\
\midrule
\cellcolor{gray!10}{ARROZ PARA SECO / Arroz de primera} & \cellcolor{gray!10}{NA} & \cellcolor{gray!10}{NA} & \cellcolor{gray!10}{NA} & \cellcolor{gray!10}{NA}\\
NA & Intercept & 0.0320478 & 0.0091674 & 0.0010582\\
\cellcolor{gray!10}{NA} & \cellcolor{gray!10}{e\_\{t-1\}} & \cellcolor{gray!10}{-0.2149072} & \cellcolor{gray!10}{0.0905619} & \cellcolor{gray!10}{0.0218769}\\
NA & Δy\_\{t-1\} & 0.2929300 & 0.0967138 & 0.0040174\\
\cellcolor{gray!10}{NA} & \cellcolor{gray!10}{Δx\_\{t-1\}} & \cellcolor{gray!10}{0.3316908} & \cellcolor{gray!10}{0.1119224} & \cellcolor{gray!10}{0.0048029}\\
\addlinespace
PAPA / Papa capira & NA & NA & NA & NA\\
\cellcolor{gray!10}{NA} & \cellcolor{gray!10}{Intercept} & \cellcolor{gray!10}{0.0895944} & \cellcolor{gray!10}{0.0487491} & \cellcolor{gray!10}{0.0725450}\\
NA & e\_\{t-1\} & -0.3879246 & 0.0877859 & 0.0000599\\
\cellcolor{gray!10}{NA} & \cellcolor{gray!10}{Δy\_\{t-1\}} & \cellcolor{gray!10}{0.3403183} & \cellcolor{gray!10}{0.1212268} & \cellcolor{gray!10}{0.0073043}\\
NA & Δx\_\{t-1\} & -0.2667158 & 0.2080773 & 0.2063322\\
\addlinespace
\cellcolor{gray!10}{PLÁTANO / Plátano hartón verde} & \cellcolor{gray!10}{NA} & \cellcolor{gray!10}{NA} & \cellcolor{gray!10}{NA} & \cellcolor{gray!10}{NA}\\
NA & Intercept & 0.0117105 & 0.0233113 & 0.6179826\\
\cellcolor{gray!10}{NA} & \cellcolor{gray!10}{e\_\{t-1\}} & \cellcolor{gray!10}{-0.1581900} & \cellcolor{gray!10}{0.0879040} & \cellcolor{gray!10}{0.0789469}\\
NA & Δy\_\{t-1\} & -0.0584523 & 0.1277311 & 0.6495294\\
\cellcolor{gray!10}{NA} & \cellcolor{gray!10}{Δy\_\{t-2\}} & \cellcolor{gray!10}{-0.1136876} & \cellcolor{gray!10}{0.1203972} & \cellcolor{gray!10}{0.3503070}\\
\addlinespace
NA & Δx\_\{t-1\} & 0.2810099 & 0.1623921 & 0.0907235\\
\cellcolor{gray!10}{NA} & \cellcolor{gray!10}{Δx\_\{t-2\}} & \cellcolor{gray!10}{0.2772650} & \cellcolor{gray!10}{0.1585799} & \cellcolor{gray!10}{0.0875273}\\
YUCA / Yuca ICA & NA & NA & NA & NA\\
\cellcolor{gray!10}{NA} & \cellcolor{gray!10}{Intercept} & \cellcolor{gray!10}{0.0733991} & \cellcolor{gray!10}{0.0224783} & \cellcolor{gray!10}{0.0020685}\\
NA & e\_\{t-1\} & -0.1847803 & 0.0567201 & 0.0021138\\
\addlinespace
\cellcolor{gray!10}{NA} & \cellcolor{gray!10}{Δy\_\{t-1\}} & \cellcolor{gray!10}{0.1698592} & \cellcolor{gray!10}{0.1505081} & \cellcolor{gray!10}{0.2649302}\\
NA & Δx\_\{t-1\} & -0.0315605 & 0.0707228 & 0.6575051\\
\bottomrule
\end{tabular}}
\end{table}

\begin{table}[!h]
\centering
\caption{\label{tab:tab:ecm_bogota}Error correction model (ECM) estimates for Bogotá. The ECM is estimated under the Enders restriction (same lag length for \(\Delta y\) and \(\Delta x\), no contemporaneous \(\Delta x_t\)).}
\centering
\resizebox{\ifdim\width>\linewidth\linewidth\else\width\fi}{!}{
\fontsize{9}{11}\selectfont
\begin{tabular}[t]{lllll}
\toprule
pair & term & b & se & p-value\\
\midrule
\cellcolor{gray!10}{ARROZ PARA SECO / Arroz de primera} & \cellcolor{gray!10}{NA} & \cellcolor{gray!10}{NA} & \cellcolor{gray!10}{NA} & \cellcolor{gray!10}{NA}\\
NA & Intercept & 0.0211261 & 0.0081628 & 0.0128737\\
\cellcolor{gray!10}{NA} & \cellcolor{gray!10}{e\_\{t-1\}} & \cellcolor{gray!10}{-0.1094686} & \cellcolor{gray!10}{0.0604712} & \cellcolor{gray!10}{0.0767906}\\
NA & Δy\_\{t-1\} & 0.3811162 & 0.0884717 & 0.0000858\\
\cellcolor{gray!10}{NA} & \cellcolor{gray!10}{Δx\_\{t-1\}} & \cellcolor{gray!10}{0.3406157} & \cellcolor{gray!10}{0.0981641} & \cellcolor{gray!10}{0.0011424}\\
\addlinespace
PAPA / Papa R-12 negra & NA & NA & NA & NA\\
\cellcolor{gray!10}{NA} & \cellcolor{gray!10}{Intercept} & \cellcolor{gray!10}{0.0641157} & \cellcolor{gray!10}{0.0345081} & \cellcolor{gray!10}{0.0695773}\\
NA & e\_\{t-1\} & -0.1753690 & 0.0614762 & 0.0064760\\
\cellcolor{gray!10}{NA} & \cellcolor{gray!10}{Δy\_\{t-1\}} & \cellcolor{gray!10}{0.5036602} & \cellcolor{gray!10}{0.1031149} & \cellcolor{gray!10}{0.0000129}\\
NA & Δx\_\{t-1\} & 0.1550001 & 0.1345910 & 0.2554203\\
\addlinespace
\cellcolor{gray!10}{PLÁTANO / Plátano hartón verde} & \cellcolor{gray!10}{NA} & \cellcolor{gray!10}{NA} & \cellcolor{gray!10}{NA} & \cellcolor{gray!10}{NA}\\
NA & Intercept & 0.0358186 & 0.0294343 & 0.2298490\\
\cellcolor{gray!10}{NA} & \cellcolor{gray!10}{e\_\{t-1\}} & \cellcolor{gray!10}{-0.2977576} & \cellcolor{gray!10}{0.0935277} & \cellcolor{gray!10}{0.0026091}\\
NA & Δy\_\{t-1\} & 0.2119419 & 0.1328323 & 0.1174355\\
\cellcolor{gray!10}{NA} & \cellcolor{gray!10}{Δx\_\{t-1\}} & \cellcolor{gray!10}{0.1607904} & \cellcolor{gray!10}{0.2152965} & \cellcolor{gray!10}{0.4589652}\\
\addlinespace
YUCA / Yuca llanera & NA & NA & NA & NA\\
\cellcolor{gray!10}{NA} & \cellcolor{gray!10}{Intercept} & \cellcolor{gray!10}{0.0291810} & \cellcolor{gray!10}{0.0235763} & \cellcolor{gray!10}{0.2220978}\\
NA & e\_\{t-1\} & -0.1983702 & 0.0783679 & 0.0148474\\
\cellcolor{gray!10}{NA} & \cellcolor{gray!10}{Δy\_\{t-1\}} & \cellcolor{gray!10}{0.1140003} & \cellcolor{gray!10}{0.1333491} & \cellcolor{gray!10}{0.3970369}\\
NA & Δx\_\{t-1\} & 0.0073602 & 0.0901300 & 0.9352695\\
\bottomrule
\end{tabular}}
\end{table}

\begin{table}[!h]
\centering
\caption{\label{tab:tab:ecm_medellin}Error correction model (ECM) estimates for Medellín. The ECM is estimated under the Enders restriction (same lag length for \(\Delta y\) and \(\Delta x\), no contemporaneous \(\Delta x_t\)).}
\centering
\resizebox{\ifdim\width>\linewidth\linewidth\else\width\fi}{!}{
\fontsize{9}{11}\selectfont
\begin{tabular}[t]{lllll}
\toprule
pair & term & b & se & p-value\\
\midrule
\cellcolor{gray!10}{ARROZ PARA SECO / Arroz de primera} & \cellcolor{gray!10}{NA} & \cellcolor{gray!10}{NA} & \cellcolor{gray!10}{NA} & \cellcolor{gray!10}{NA}\\
NA & Intercept & 0.0298102 & 0.0106044 & 0.0072308\\
\cellcolor{gray!10}{NA} & \cellcolor{gray!10}{e\_\{t-1\}} & \cellcolor{gray!10}{-0.1594146} & \cellcolor{gray!10}{0.0694176} & \cellcolor{gray!10}{0.0262540}\\
NA & Δy\_\{t-1\} & 0.3516071 & 0.1018648 & 0.0012051\\
\cellcolor{gray!10}{NA} & \cellcolor{gray!10}{Δx\_\{t-1\}} & \cellcolor{gray!10}{0.3294578} & \cellcolor{gray!10}{0.1436373} & \cellcolor{gray!10}{0.0264265}\\
\addlinespace
PAPA / Papa capira & NA & NA & NA & NA\\
\cellcolor{gray!10}{NA} & \cellcolor{gray!10}{Intercept} & \cellcolor{gray!10}{0.1524624} & \cellcolor{gray!10}{0.0388060} & \cellcolor{gray!10}{0.0002845}\\
NA & e\_\{t-1\} & -0.4447105 & 0.0857008 & 0.0000046\\
\cellcolor{gray!10}{NA} & \cellcolor{gray!10}{Δy\_\{t-1\}} & \cellcolor{gray!10}{0.3041604} & \cellcolor{gray!10}{0.1085970} & \cellcolor{gray!10}{0.0074302}\\
NA & Δx\_\{t-1\} & -0.3303416 & 0.2204563 & 0.1408494\\
\addlinespace
\cellcolor{gray!10}{PLÁTANO / Plátano hartón verde} & \cellcolor{gray!10}{NA} & \cellcolor{gray!10}{NA} & \cellcolor{gray!10}{NA} & \cellcolor{gray!10}{NA}\\
NA & Intercept & 0.0061420 & 0.0241712 & 0.8005463\\
\cellcolor{gray!10}{NA} & \cellcolor{gray!10}{e\_\{t-1\}} & \cellcolor{gray!10}{-0.1026900} & \cellcolor{gray!10}{0.0511828} & \cellcolor{gray!10}{0.0507202}\\
NA & Δy\_\{t-1\} & 0.1281623 & 0.1315426 & 0.3350022\\
\cellcolor{gray!10}{NA} & \cellcolor{gray!10}{Δx\_\{t-1\}} & \cellcolor{gray!10}{0.2369336} & \cellcolor{gray!10}{0.1336452} & \cellcolor{gray!10}{0.0828742}\\
\addlinespace
YUCA / Yuca ICA & NA & NA & NA & NA\\
\cellcolor{gray!10}{NA} & \cellcolor{gray!10}{Intercept} & \cellcolor{gray!10}{0.0290916} & \cellcolor{gray!10}{0.0190291} & \cellcolor{gray!10}{0.1331632}\\
NA & e\_\{t-1\} & -0.1737341 & 0.0642461 & 0.0095656\\
\cellcolor{gray!10}{NA} & \cellcolor{gray!10}{Δy\_\{t-1\}} & \cellcolor{gray!10}{0.2168052} & \cellcolor{gray!10}{0.1246585} & \cellcolor{gray!10}{0.0886895}\\
NA & Δx\_\{t-1\} & 0.1433950 & 0.0725168 & 0.0540063\\
\bottomrule
\end{tabular}}
\end{table}

\subsection{Análisis de cointegración
asimétrica}\label{anuxe1lisis-de-cointegraciuxf3n-asimuxe9trica}

\subsubsection{Modelo de Corrección de Error Asimétrico
(A-ECM)}\label{modelo-de-correcciuxf3n-de-error-asimuxe9trico-a-ecm-1}

\subsubsection{Modelo TAR}\label{modelo-tar}

\subsubsection{Modelo M-TAR}\label{modelo-m-tar}

  \bibliography{refs-ob-rif-informal.bib}

\end{document}
