% Options for packages loaded elsewhere
\PassOptionsToPackage{unicode}{hyperref}
\PassOptionsToPackage{hyphens}{url}
\documentclass[
]{article}
\usepackage{xcolor}
\usepackage[margin=2.5cm]{geometry}
\usepackage{amsmath,amssymb}
\setcounter{secnumdepth}{-\maxdimen} % remove section numbering
\usepackage{iftex}
\ifPDFTeX
  \usepackage[T1]{fontenc}
  \usepackage[utf8]{inputenc}
  \usepackage{textcomp} % provide euro and other symbols
\else % if luatex or xetex
  \usepackage{unicode-math} % this also loads fontspec
  \defaultfontfeatures{Scale=MatchLowercase}
  \defaultfontfeatures[\rmfamily]{Ligatures=TeX,Scale=1}
\fi
\usepackage{lmodern}
\ifPDFTeX\else
  % xetex/luatex font selection
\fi
% Use upquote if available, for straight quotes in verbatim environments
\IfFileExists{upquote.sty}{\usepackage{upquote}}{}
\IfFileExists{microtype.sty}{% use microtype if available
  \usepackage[]{microtype}
  \UseMicrotypeSet[protrusion]{basicmath} % disable protrusion for tt fonts
}{}
\makeatletter
\@ifundefined{KOMAClassName}{% if non-KOMA class
  \IfFileExists{parskip.sty}{%
    \usepackage{parskip}
  }{% else
    \setlength{\parindent}{0pt}
    \setlength{\parskip}{6pt plus 2pt minus 1pt}}
}{% if KOMA class
  \KOMAoptions{parskip=half}}
\makeatother
\usepackage{graphicx}
\makeatletter
\newsavebox\pandoc@box
\newcommand*\pandocbounded[1]{% scales image to fit in text height/width
  \sbox\pandoc@box{#1}%
  \Gscale@div\@tempa{\textheight}{\dimexpr\ht\pandoc@box+\dp\pandoc@box\relax}%
  \Gscale@div\@tempb{\linewidth}{\wd\pandoc@box}%
  \ifdim\@tempb\p@<\@tempa\p@\let\@tempa\@tempb\fi% select the smaller of both
  \ifdim\@tempa\p@<\p@\scalebox{\@tempa}{\usebox\pandoc@box}%
  \else\usebox{\pandoc@box}%
  \fi%
}
% Set default figure placement to htbp
\def\fps@figure{htbp}
\makeatother
\setlength{\emergencystretch}{3em} % prevent overfull lines
\providecommand{\tightlist}{%
  \setlength{\itemsep}{0pt}\setlength{\parskip}{0pt}}
\usepackage{bookmark}
\IfFileExists{xurl.sty}{\usepackage{xurl}}{} % add URL line breaks if available
\urlstyle{same}
\hypersetup{
  pdftitle={Anexos técnicos},
  pdfauthor={Pontificia Universidad Javeriana, Cali},
  hidelinks,
  pdfcreator={LaTeX via pandoc}}

\title{Anexos técnicos}
\usepackage{etoolbox}
\makeatletter
\providecommand{\subtitle}[1]{% add subtitle to \maketitle
  \apptocmd{\@title}{\par {\large #1 \par}}{}{}
}
\makeatother
\subtitle{Documento de trabajo 1225}
\author{Pontificia Universidad Javeriana, Cali}
\date{14 de diciembre de 2025}

\begin{document}
\maketitle

\section{Anexos técnicos}\label{anexos-tuxe9cnicos}

En esta sección se presentan los anexos técnicos que complementan el
análisis principal del documento. Cada anexo desarrolla en detalle los
resultados empíricos, tablas y representaciones gráficas asociadas a las
metodologías implementadas para la estimación y validación de los
precios minoristas de los alimentos.

\subsection{Anexo A. Metodología I: Aproximación a partir del
IPC}\label{anexo-a.-metodologuxeda-i-aproximaciuxf3n-a-partir-del-ipc}

Este anexo presenta los resultados completos de la metodología basada en
la variación mensual del IPC a nivel de subclase. Se incluyen las
comparaciones entre precios reales y precios estimados para todas las
subclases y ciudades analizadas, así como las métricas de desempeño
correspondientes.

\includegraphics[width=1\linewidth]{m0/output_ipc_fill/plots_grouped/CALI__page_01}
\includegraphics[width=1\linewidth]{m0/output_ipc_fill/plots_grouped/CALI__page_02}
\includegraphics[width=1\linewidth]{m0/output_ipc_fill/plots_grouped/CALI__page_03}
\includegraphics[width=1\linewidth]{m0/output_ipc_fill/plots_grouped/CALI__page_04}
\includegraphics[width=1\linewidth]{m0/output_ipc_fill/plots_grouped/CALI__page_05}
\includegraphics[width=1\linewidth]{m0/output_ipc_fill/plots_grouped/CALI__page_06}
\includegraphics[width=1\linewidth]{m0/output_ipc_fill/plots_grouped/CALI__page_07}
\includegraphics[width=1\linewidth]{m0/output_ipc_fill/plots_grouped/CALI__page_08}
\includegraphics[width=1\linewidth]{m0/output_ipc_fill/plots_grouped/CALI__page_09}
\includegraphics[width=1\linewidth]{m0/output_ipc_fill/plots_grouped/CALI__page_10}
\includegraphics[width=1\linewidth]{m0/output_ipc_fill/plots_grouped/CALI__page_11}
\includegraphics[width=1\linewidth]{m0/output_ipc_fill/plots_grouped/CALI__page_12}

\subsection{Anexo B. Metodología II: Estimación a partir del margen
mediano (Q1 --
Q3)}\label{anexo-b.-metodologuxeda-ii-estimaciuxf3n-a-partir-del-margen-mediano-q1-q3}

Este anexo documenta los resultados obtenidos a partir del cálculo de
los márgenes de comercialización implícitos, incluyendo la distribución
de los márgenes y los percentiles utilizados para la estimación de
precios minoristas.

\includegraphics[width=1\linewidth]{m1/output/121225_m1_forecast_by_food_page_01}
\includegraphics[width=1\linewidth]{m1/output/121225_m1_forecast_by_food_page_02}
\includegraphics[width=1\linewidth]{m1/output/121225_m1_forecast_by_food_page_03}
\includegraphics[width=1\linewidth]{m1/output/121225_m1_forecast_by_food_page_04}
\includegraphics[width=1\linewidth]{m1/output/121225_m1_forecast_by_food_page_05}
\includegraphics[width=1\linewidth]{m1/output/121225_m1_forecast_by_food_page_06}
\includegraphics[width=1\linewidth]{m1/output/121225_m1_forecast_by_food_page_07}
\includegraphics[width=1\linewidth]{m1/output/121225_m1_forecast_by_food_page_08}
\includegraphics[width=1\linewidth]{m1/output/121225_m1_forecast_by_food_page_09}

\subsection{Anexo C. Metodología III: Regresión lineal en niveles y
primeras
diferencias}\label{anexo-c.-metodologuxeda-iii-regresiuxf3n-lineal-en-niveles-y-primeras-diferencias}

En este anexo se presentan los resultados detallados de los modelos de
regresión lineal estimados en niveles logarítmicos, incorporando efectos
estacionales mediante variables dummy mensuales.

\includegraphics[width=1\linewidth]{m6/output_dummies/m6_dummies_grouped_page_01}
\includegraphics[width=1\linewidth]{m6/output_dummies/m6_dummies_grouped_page_02}
\includegraphics[width=1\linewidth]{m6/output_dummies/m6_dummies_grouped_page_03}
\includegraphics[width=1\linewidth]{m6/output_dummies/m6_dummies_grouped_page_04}
\includegraphics[width=1\linewidth]{m6/output_dummies/m6_dummies_grouped_page_05}
\includegraphics[width=1\linewidth]{m6/output_dummies/m6_dummies_grouped_page_06}
\includegraphics[width=1\linewidth]{m6/output_dummies/m6_dummies_grouped_page_07}
\includegraphics[width=1\linewidth]{m6/output_dummies/m6_dummies_grouped_page_08}
\includegraphics[width=1\linewidth]{m6/output_dummies/m6_dummies_grouped_page_09}

\subsection{Anexo D. Metodología IV: Regresión lineal en primeras
diferencias}\label{anexo-d.-metodologuxeda-iv-regresiuxf3n-lineal-en-primeras-diferencias}

Este anexo recoge los resultados de la estimación en primeras
diferencias, aplicada para evitar problemas de no estacionariedad y
correlaciones espurias, conforme a las pruebas de raíz unitaria
realizadas.

\includegraphics[width=1\linewidth]{m7/output_dummies/m7_dummies_grouped_page_01}
\includegraphics[width=1\linewidth]{m7/output_dummies/m7_dummies_grouped_page_02}
\includegraphics[width=1\linewidth]{m7/output_dummies/m7_dummies_grouped_page_03}
\includegraphics[width=1\linewidth]{m7/output_dummies/m7_dummies_grouped_page_04}
\includegraphics[width=1\linewidth]{m7/output_dummies/m7_dummies_grouped_page_05}
\includegraphics[width=1\linewidth]{m7/output_dummies/m7_dummies_grouped_page_06}
\includegraphics[width=1\linewidth]{m7/output_dummies/m7_dummies_grouped_page_07}
\includegraphics[width=1\linewidth]{m7/output_dummies/m7_dummies_grouped_page_08}

\subsection{Anexo E. Metodología V: Modelo de Corrección de Error
(ECM)}\label{anexo-e.-metodologuxeda-v-modelo-de-correcciuxf3n-de-error-ecm}

Este anexo desarrolla en detalle los resultados del modelo de corrección
de error, incluyendo la relación de largo plazo entre precios mayoristas
y minoristas, así como la dinámica de ajuste de corto plazo.

\includegraphics[width=1\linewidth]{m5/output_ecm/m5_ecm_grouped_page_01}
\includegraphics[width=1\linewidth]{m5/output_ecm/m5_ecm_grouped_page_02}
\includegraphics[width=1\linewidth]{m5/output_ecm/m5_ecm_grouped_page_03}
\includegraphics[width=1\linewidth]{m5/output_ecm/m5_ecm_grouped_page_04}
\includegraphics[width=1\linewidth]{m5/output_ecm/m5_ecm_grouped_page_05}
\includegraphics[width=1\linewidth]{m5/output_ecm/m5_ecm_grouped_page_06}
\includegraphics[width=1\linewidth]{m5/output_ecm/m5_ecm_grouped_page_07}
\includegraphics[width=1\linewidth]{m5/output_ecm/m5_ecm_grouped_page_08}

\subsection{Anexo F. Metodología VI: Modelo de Corrección de Error
Asimétrico
(A-ECM)}\label{anexo-f.-metodologuxeda-vi-modelo-de-correcciuxf3n-de-error-asimuxe9trico-a-ecm}

Este anexo presenta los resultados del modelo ECM asimétrico, diseñado
para capturar posibles diferencias en la velocidad y el patrón de ajuste
ante aumentos y reducciones en el precio mayorista.

\includegraphics[width=1\linewidth]{m4/output/m4_ecm_grouped_page_01}
\includegraphics[width=1\linewidth]{m4/output/m4_ecm_grouped_page_02}
\includegraphics[width=1\linewidth]{m4/output/m4_ecm_grouped_page_03}
\includegraphics[width=1\linewidth]{m4/output/m4_ecm_grouped_page_04}
\includegraphics[width=1\linewidth]{m4/output/m4_ecm_grouped_page_05}
\includegraphics[width=1\linewidth]{m4/output/m4_ecm_grouped_page_06}
\includegraphics[width=1\linewidth]{m4/output/m4_ecm_grouped_page_07}
\includegraphics[width=1\linewidth]{m4/output/m4_ecm_grouped_page_08}

\end{document}
